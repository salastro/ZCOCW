\documentclass{article}
%%%%%%%%%%%%%%%%%%%%%%%%%%%%%%%%%%%%%%%%%%%%%%%%%%%%%%%%%%%%%%%%%%%%%%%%%%%%%%%
%                                Basic Packages                               %
%%%%%%%%%%%%%%%%%%%%%%%%%%%%%%%%%%%%%%%%%%%%%%%%%%%%%%%%%%%%%%%%%%%%%%%%%%%%%%%

% Lets us style link.
\usepackage{xcolor}
\usepackage{hyperref}
% Lets us import images and graphics.
\usepackage{graphicx}
% Gives us better math syntax.
\usepackage{amsmath,amsfonts,mathtools,amsthm,amssymb, scalerel, bm}
\usepackage{accents}
\usepackage{multicol}
% Lets us import pdf directly in our tex code.
\usepackage{pdfpages}
% Lets us edit the caption of a figure.
\usepackage{caption}

%%%%%%%%%%%%%%%%%%%%%%%%%%%%%%%%%%%%%%%%%%%%%%%%%%%%%%%%%%%%%%%%%%%%%%%%%%%%%%%
%                                Basic Settings                               %
%%%%%%%%%%%%%%%%%%%%%%%%%%%%%%%%%%%%%%%%%%%%%%%%%%%%%%%%%%%%%%%%%%%%%%%%%%%%%%%

%%%%%%%%%%%%%%%%%%%%
%  Document setup  %
%%%%%%%%%%%%%%%%%%%%

\usepackage[a4paper, margin=1.25in]{geometry}
\usepackage[UST, 5, OnlyFrontpage]{mnfrontpage}

%%%%%%%%%%%%%%%%%%%%%
%  Useful commands  %
%%%%%%%%%%%%%%%%%%%%%

\newcommand{\hh}{ \vspace{1em}\hrule\vspace{1em} }

%%%%%%%%%%%%%%%%
%  Set indent  %
%%%%%%%%%%%%%%%%

\newlength\tindent
\setlength{\tindent}{\parindent}
\setlength{\parindent}{0pt}
\renewcommand{\indent}{\hspace*{\tindent}}

%%%%%%%%%%%%%
%  Symbols  %
%%%%%%%%%%%%%

\let\implies\Rightarrow
\let\impliedby\Leftarrow
\let\iff\Leftrightarrow
% \let\epsilon\varepsilon

%%%%%%%%%%%%%%
%  SI Unitx  %
%%%%%%%%%%%%%%

\usepackage{siunitx}
\sisetup{locale = FR}

%%%%%%%%%%
%  TikZ  %
%%%%%%%%%%

\usepackage[framemethod=TikZ]{mdframed}
\usepackage{tikz}
\usepackage{tikz-cd}
\usepackage{tikzsymbols}

\usetikzlibrary{intersections, angles, quotes, calc, positioning}
\usetikzlibrary{arrows.meta}

\tikzset{
    force/.style={thick, {Circle[length=2pt]}-stealth, shorten <=-1pt}
}

%%%%%%%%%%%%%%%
%  PGF Plots  %
%%%%%%%%%%%%%%%

\usepackage{pgfplots}
\pgfplotsset{compat=1.13}

%%%%%%%%%%%%%%%%%%%%%%%%
%  Modify Links Color  %
%%%%%%%%%%%%%%%%%%%%%%%%

\hypersetup{
    % Enable highlighting links.
    colorlinks,
    % Change the color of links to blue.
    linkcolor=blue,
    % Change the color of citations to black.
    citecolor={black},
    % Change the color of url's to blue with some black.
    urlcolor={blue!80!black}
}

%%%%%%%%%%%%%%%%%%%
%  Todo Commands  %
%%%%%%%%%%%%%%%%%%%

\usepackage{xargs}
\usepackage[colorinlistoftodos]{todonotes}

\newcommandx\unsure[2][1=]{\todo[linecolor=red,backgroundcolor=red!25,bordercolor=red,#1]{#2}}
\newcommandx\change[2][1=]{\todo[linecolor=blue,backgroundcolor=blue!25,bordercolor=blue,#1]{#2}}
\newcommandx\info[2][1=]{\todo[linecolor=green,backgroundcolor=green!25,bordercolor=green,#1]{#2}}
\newcommandx\improvement[2][1=]{\todo[linecolor=purple,backgroundcolor=purple!25,bordercolor=purple,#1]{#2}}

%%%%%%%%%%%%%%%%%%%%%%%%%%%%%%%%%%%%%%%%%%%%%%%%%%%%%%%%%%%%%%%%%%%%%%%%%%%%%%%
%                                 Environments                                %
%%%%%%%%%%%%%%%%%%%%%%%%%%%%%%%%%%%%%%%%%%%%%%%%%%%%%%%%%%%%%%%%%%%%%%%%%%%%%%%

\usepackage{varwidth}
\usepackage{thmtools}
\usepackage[most,many,breakable]{tcolorbox}

\tcbuselibrary{theorems,skins,hooks}
\usetikzlibrary{arrows,calc,shadows.blur}

%%%%%%%%%%%%%%%%%%%
%  Define Colors  %
%%%%%%%%%%%%%%%%%%%

\definecolor{myblue}{RGB}{45, 111, 177}
\definecolor{mygreen}{RGB}{56, 140, 70}
\definecolor{myred}{RGB}{199, 68, 64}
\definecolor{mypurple}{RGB}{197, 92, 212}

\definecolor{definition}{HTML}{228b22}
\definecolor{theorem}{HTML}{00007B}
\definecolor{example}{HTML}{2A7F7F}
\definecolor{definition}{HTML}{228b22}
\definecolor{prop}{HTML}{191971}
\definecolor{lemma}{HTML}{983b0f}
\definecolor{exercise}{HTML}{88D6D1}

\colorlet{definition}{mygreen!85!black}
\colorlet{claim}{mygreen!85!black}
\colorlet{corollary}{mypurple!85!black}
\colorlet{proof}{theorem}

%%%%%%%%%%%%%%%%%%%%%%%%%%%%%%%%%%%%%%%%%%%%%%%%%%%%%%%%%
%  Create Environments Styles Based on Given Parameter  %
%%%%%%%%%%%%%%%%%%%%%%%%%%%%%%%%%%%%%%%%%%%%%%%%%%%%%%%%%

\mdfsetup{skipabove=1em,skipbelow=0em}

%%%%%%%%%%%%%%%%%%%%%%
%  Helpful Commands  %
%%%%%%%%%%%%%%%%%%%%%%

% EXAMPLE:
% 1. \createnewtheoremstyle{thmdefinitionbox}{}{}
% 2. \createnewtheoremstyle{thmtheorembox}{}{}
% 3. \createnewtheoremstyle{thmproofbox}{qed=\qedsymbol}{
%       rightline=false, topline=false, bottomline=false
%    }
% Parameters:
% 1. Theorem name.
% 2. Any extra parameters to pass directly to declaretheoremstyle.
% 3. Any extra parameters to pass directly to mdframed.
\newcommand\createnewtheoremstyle[3]{
    \declaretheoremstyle[
    headfont=\bfseries\sffamily, bodyfont=\normalfont, #2,
    mdframed={
        #3,
    },
    ]{#1}
}

% EXAMPLE:
% 1. \createnewcoloredtheoremstyle{thmdefinitionbox}{definition}{}{}
% 2. \createnewcoloredtheoremstyle{thmexamplebox}{example}{}{
%       rightline=true, leftline=true, topline=true, bottomline=true
%     }
% 3. \createnewcoloredtheoremstyle{thmproofbox}{proof}{qed=\qedsymbol}{backgroundcolor=white}
% Parameters:
% 1. Theorem name.
% 2. Color of theorem.
% 3. Any extra parameters to pass directly to declaretheoremstyle.
% 4. Any extra parameters to pass directly to mdframed.
\newcommand\createnewcoloredtheoremstyle[4]{
    \declaretheoremstyle[
    headfont=\bfseries\sffamily\color{#2}, bodyfont=\normalfont, #3,
    mdframed={
        linewidth=2pt,
        rightline=false, leftline=true, topline=false, bottomline=false,
        linecolor=#2, backgroundcolor=#2!5, #4,
    },
    ]{#1}
}

%%%%%%%%%%%%%%%%%%%%%%%%%%%%%%%%%%%
%  Create the Environment Styles  %
%%%%%%%%%%%%%%%%%%%%%%%%%%%%%%%%%%%

\createnewcoloredtheoremstyle{thmdefinitionbox}{definition}{}{}
\createnewcoloredtheoremstyle{thmtheorembox}{theorem}{}{}
\createnewcoloredtheoremstyle{thmexamplebox}{example}{}{
    rightline=true, leftline=true, topline=true, bottomline=true
}
\createnewcoloredtheoremstyle{thmclaimbox}{claim}{}{}
\createnewcoloredtheoremstyle{thmcorollarybox}{corollary}{}{}
\createnewcoloredtheoremstyle{thmpropbox}{prop}{}{}
\createnewcoloredtheoremstyle{thmlemmabox}{lemma}{}{}
\createnewcoloredtheoremstyle{thmexercisebox}{exercise}{}{}

\createnewcoloredtheoremstyle{thmproofbox}{proof}{qed=\qedsymbol}{backgroundcolor=white}
\createnewcoloredtheoremstyle{thmexplanationbox}{example}{}{backgroundcolor=white}

%%%%%%%%%%%%%%%%%%%%%%%%%%%%%
%  Create the Environments  %
%%%%%%%%%%%%%%%%%%%%%%%%%%%%%

\declaretheorem[numberwithin=section, style=thmtheorembox,     name=Theorem]{theorem}
\declaretheorem[numbered=no,          style=thmexamplebox,     name=Example]{example}
\declaretheorem[numberwithin=section, style=thmclaimbox,       name=Claim]{claim}
\declaretheorem[numberwithin=section, style=thmcorollarybox,   name=Corollary]{corollary}
\declaretheorem[numberwithin=section, style=thmpropbox,        name=Proposition]{prop}
\declaretheorem[numberwithin=section, style=thmlemmabox,       name=Lemma]{lemma}
\declaretheorem[numberwithin=section, style=thmexercisebox,    name=Exercise]{exercise}
\declaretheorem[numbered=no,          style=thmproofbox,       name=Proof]{replacementproof}
\declaretheorem[numbered=no,          style=thmexplanationbox, name=Conclusion]{conc}

\newtcbtheorem[number within=section]{Definition}{Definition}{
    enhanced,
    before skip=2mm,
    after skip=2mm,
    colback=red!5,
    colframe=red!80!black,
    colbacktitle=red!75!black,
    boxrule=0.5mm,
    attach boxed title to top left={
        xshift=1cm,
        yshift*=1mm-\tcboxedtitleheight
    },
    varwidth boxed title*=-3cm,
    boxed title style={
        interior engine=empty,
        frame code={
            \path[fill=tcbcolback]
            ([yshift=-1mm,xshift=-1mm]frame.north west)
            arc[start angle=0,end angle=180,radius=1mm]
            ([yshift=-1mm,xshift=1mm]frame.north east)
            arc[start angle=180,end angle=0,radius=1mm];
            \path[left color=tcbcolback!60!black,right color=tcbcolback!60!black,
            middle color=tcbcolback!80!black]
            ([xshift=-2mm]frame.north west) -- ([xshift=2mm]frame.north east)
            [rounded corners=1mm]-- ([xshift=1mm,yshift=-1mm]frame.north east)
            -- (frame.south east) -- (frame.south west)
            -- ([xshift=-1mm,yshift=-1mm]frame.north west)
            [sharp corners]-- cycle;
        },
    },
    fonttitle=\bfseries,
    title={#2},
    #1
}{def}

\NewDocumentEnvironment{definition}{O{}O{}}
{\begin{Definition}{#1}{#2}}{\end{Definition}}

\newtcolorbox{note}[1][]{%
    enhanced jigsaw,
    colback=gray!20!white,%
    colframe=gray!80!black,
    size=small,
    boxrule=1pt,
    title=\textbf{Note:-},
    halign title=flush center,
    coltitle=black,
    breakable,
    drop shadow=black!50!white,
    attach boxed title to top left={xshift=1cm,yshift=-\tcboxedtitleheight/2,yshifttext=-\tcboxedtitleheight/2},
    minipage boxed title=1.5cm,
    boxed title style={%
        colback=white,
        size=fbox,
        boxrule=1pt,
        boxsep=2pt,
        underlay={%
            \coordinate (dotA) at ($(interior.west) + (-0.5pt,0)$);
            \coordinate (dotB) at ($(interior.east) + (0.5pt,0)$);
            \begin{scope}
                \clip (interior.north west) rectangle ([xshift=3ex]interior.east);
                \filldraw [white, blur shadow={shadow opacity=60, shadow yshift=-.75ex}, rounded corners=2pt] (interior.north west) rectangle (interior.south east);
            \end{scope}
            \begin{scope}[gray!80!black]
                \fill (dotA) circle (2pt);
                \fill (dotB) circle (2pt);
            \end{scope}
        },
    },
    #1,
}

\newtcbtheorem{Question}{Question}{enhanced,
    breakable,
    colback=white,
    colframe=myblue!80!black,
    attach boxed title to top left={yshift*=-\tcboxedtitleheight},
    fonttitle=\bfseries,
    title=\textbf{Question:-},
    boxed title size=title,
    boxed title style={%
        sharp corners,
        rounded corners=northwest,
        colback=tcbcolframe,
        boxrule=0pt,
    },
    underlay boxed title={%
        \path[fill=tcbcolframe] (title.south west)--(title.south east)
        to[out=0, in=180] ([xshift=5mm]title.east)--
        (title.center-|frame.east)
        [rounded corners=\kvtcb@arc] |-
        (frame.north) -| cycle;
    },
    #1
}{def}

\NewDocumentEnvironment{question}{O{}O{}}
{\begin{Question}{#1}{#2}}{\end{Question}}

\newtcolorbox{Solution}{enhanced,
    breakable,
    colback=white,
    colframe=mygreen!80!black,
    attach boxed title to top left={yshift*=-\tcboxedtitleheight},
    title=\textbf{Solution:-},
    boxed title size=title,
    boxed title style={%
        sharp corners,
        rounded corners=northwest,
        colback=tcbcolframe,
        boxrule=0pt,
    },
    underlay boxed title={%
        \path[fill=tcbcolframe] (title.south west)--(title.south east)
        to[out=0, in=180] ([xshift=5mm]title.east)--
        (title.center-|frame.east)
        [rounded corners=\kvtcb@arc] |-
        (frame.north) -| cycle;
    },
}

\NewDocumentEnvironment{solution}{O{}O{}}
{\vspace{-10pt}\begin{Solution}{#1}{#2}}{\end{Solution}}

%%%%%%%%%%%%%%%%%%%%%%%%%%%%
%  Edit Proof Environment  %
%%%%%%%%%%%%%%%%%%%%%%%%%%%%

\renewenvironment{proof}[1][\proofname]{\vspace{-10pt}\begin{replacementproof}}{\end{replacementproof}}
\newenvironment{explanation}[1][\proofname]{\vspace{-10pt}\begin{expl}}{\end{expl}}

\theoremstyle{definition}

\newtheorem*{notation}{Notation}
\newtheorem*{previouslyseen}{As previously seen}
\newtheorem*{problem}{Problem}
\newtheorem*{observe}{Observe}
\newtheorem*{property}{Property}
\newtheorem*{intuition}{Intuition}


% Title
\title{MATH 201}
\subtitle{Linear Algebra and Vector Geometry}
\author{Open Courseware}
\kind{Lecture \#1 \hfill \today}
\date{\today}

% Document
\begin{document}

% front matter
\mnfrontpage
\tableofcontents

% main matter
\section{Lec 1}

\begin{definition}[Linear Equation]
   It is an equation where all the variables have a power of 1, with the general formula $a_1x_1+a_2x_2+a_3x_3.....=b$.
\end{definition}

\begin{example}
    \[
       x=y; y=2x+3
    .\] 
\end{example}

It is called a linear equation because, in 2-D, its solution is a point of intersection of two straight lines. and the two variables it can have are x and y. While in 3-D, the Solution is a plane produced by the intersections of straight lines. 

\begin{definition}[A system of linear equations ]
    It is a group of linear equations with
the Same Shared Solution, which is solved simultaneously. 
\end{definition}
\begin{center}

$\begin{cases}
    a_{11}x_1+a_{12}x_2+a_{13}x_3.....=b_1\\
  a_{21}x_1+a_{22}x_2+a_{23}x_3.....=b_2\\
  a_{31}x_1+a_{32}x_2+a_{33}x_3.....=b_3
\end{cases}$
  
\end{center}
here is a system of three equations of three variables, more equations, and variables can be included in one system of equations.
\begin{example}
   \[
      x+2y=5 
      .\]
      \[
      x+3y=8 
      .\]
    \[
     x-x+2y-3y=8-5 \Rightarrow y=3      
    \]      
      solving in one of the first two equations gives $x=-1$
\end{example}

we can say that 

\[
   \begin{cases}
        x+2y=5 \\ 
        y=3
    \end{cases} 
    and  \hspace{1cm} 
    \begin{cases} 
        x+2y=5 \\ 
        x+3y=8 
    \end{cases} \\
.\]
are a row equivalent system, which means they can be changed into each other by elementary operations. 

\subsection{Representing a system of linear equations in a matrix }

$\begin{cases}
    a_{11}x_1+a_{12}x_2+.....a_{1n}x_n=b_1\\
    a_{21}x_1+a_{22}x_2+.....a_{2n}x_n=b_2\\
    a_{n1}x_1+a_{n2}x_2+.....a_{nn}x_n=b_n 
\end{cases} \Rightarrow
\begin{bmatrix}   
            a_{11} & a_{12} & \cdots & a_{1n} & b_1 \\
            a_{21} & a_{22} & \cdots & a_{2n} & b_2 \\
            \vdots & \ddots & \vdots & \vdots & \vdots \\
            a_{n1} & a_{n2} & \cdots & a_{nn} & \cdots b_n
\end{bmatrix}$

in the last raw of the matrix  $a_{n1} ,a_{n2} , a_{nn}$ represent the coefficient of the matrix.
\begin{example}
    \[
         \begin{cases}
            2x+y+z=1\\ -y-2x=-4\\ -4x=-4
         \end{cases} 
    \Rightarrow
            \begin{bmatrix}
                 2&1&1&1\\
                -2&-1&0&-4\\
                -4&0&0&-4\\
            \end{bmatrix} 
   .\]
\end{example}
We can perform a number of operations on a matrix that will also not affect the solution.
               \begin{enumerate}   
                   \item Multiplication or division of any row by another one. 
                   \item Adding any 2 rows to each other.
                   \item Interchanging any rows.
               \end{enumerate}
\begin{note}
    this is called Gauss elimination with back substitution.
\end{note}

 \begin{example}[system of two variables]
    \[
         \begin{cases}
            2x+y=4\\ x+2y=5
          \end{cases} \Rightarrow
          \begin{cases}
            x+2y=5\\ 2x+y=4 
          \end{cases}\Rightarrow
          \begin{cases}
            -2x-4y=-10\\ 2x+y=4 
          \end{cases}\Rightarrow
          \begin{cases}
            x+2y=5\\ -3y=-6
          \end{cases}
    \]
 \end{example}            
 Now for a bigger system
 \[
   \begin{cases}
        2x+y+z=1\\ 6x+2y+z=-1 \\ -2x+2y+z=7
   \end{cases} \Rightarrow 
   \begin{bmatrix}
        2&1&1&1\\ 6&2&1&-1\\ -2&2&1&7
   \end{bmatrix}
 \]
We want to reach something like this:
\begin{center}
     $2x+y+z=1$\\ 
   $()y+()z=()$\textrightarrow no x\\
   or $()z=()$ \textrightarrow no x or y 
\end{center}
So we keep x in the 1st equation and eliminate it from the rest, then we keep y in the 2nd equation and the 1st and eliminate it from the 3rd, and so on... 
\begin{example}
    \[
    1
    \begin{bmatrix}
        2&1&1&1\\ 6&2&1&-1\\ -2&2&1&7
    \end{bmatrix} \xrightarrow{R_1x-3+R_2}
    \begin{bmatrix}
        2&1&1&1\\ 0&-1&-2&-4\\ -2&2&1&7
    \end{bmatrix}\\
    \]
to eliminate 6 we multiply the 1st row by -3 and add it to the 2nd one
     \[
     2
      \begin{bmatrix}
        2&1&1&1\\ 0&-1&-2&-4\\ -2&2&1&7
      \end{bmatrix}\xrightarrow{R_1+R_3}
      \begin{bmatrix}
          2&1&1&1\\ 0&-1&-2&-4\\ 0&3&2&8
      \end{bmatrix}\\
     \]
     \[
     3
     \begin{bmatrix}
          2&1&1&1\\ 0&-1&-2&-4\\ 0&3&2&8
      \end{bmatrix}\xrightarrow{R_2x_3+R_3}
      \begin{bmatrix}
           2&1&1&1\\ 0&-1&-2&-4\\ 0&0&-4&-4
      \end{bmatrix} \rightarrow \text{This is called the echelon form }
     \]
so, we can now write it as
\begin{align*}
        2x+y+z=1\\
        -y-2z=-4 \Rightarrow y=4-2z\\ 
        -4z=-4 \Rightarrow z=1
\end{align*}
by substitution, we can get the solution which is:
\[ x=-1  ;  y=2   ;  z=1 \]
\end{example}

\begin{note}
the echelon form should look like this:\\
    $\begin{bmatrix}
        \blacksquare & \square & \square\\
        \triangle & \blacksquare & \square\\
        \triangle & \triangle &   \blacksquare
    \end{bmatrix}$ \textrightarrow where $\blacksquare$ is any non-zero(pivot) and  $\square$ is any number.
\end{note}
\begin{note}
    to check your answer substitute in all equations of the system. 
\end{note}
Now let's see if every system of linear equations has a solution and if it does is it a unique one? 
\begin{enumerate}
\item
    \[
    \begin{cases}
        x_1-2x_2=-1\\-x_1+3x_2=3
    \end{cases}\Rightarrow 
    \begin{bmatrix}
        1&-2&-1\\ -1&3&3
    \end{bmatrix} \xrightarrow{R
    -1+R_2}
    \begin{bmatrix}
        1&-2&-1\\ 0&1&2
    \end{bmatrix}
    \]
so, \[x_1=3 ; X_2=2\]
this system has a unique solution that represents a point of intersection of the two lines that are represented by the system's equations.  
\item
    \[
    \begin{cases}
        x_1-2x_2=-1\\-x_1+2x_2=1
    \end{cases}\Rightarrow \text{these are two coincident lines}
    \]
so the solution will be the line this system represents\textrightarrow infinite points \textrightarrow infinite solutions.\\
For the solution, we write the general form $x_1=2x_2+1$

\item If the two equations of the system represent two parallel lines then there is no solution 
    \[
     \begin{cases}
        x_1-2x_1=-1\\-x_1+2x_1=3
    \end{cases} \Rightarrow \text{parallel lines} 
    \]
\begin{align}
    \begin{bmatrix}
        1&-2&1\\ -1&2&3 
    \end{bmatrix}\xrightarrow{R_1+R_2}
    \begin{bmatrix}
         1&-2&1\\ 0&0&2
    \end{bmatrix} \rightarrow x_1 +2x_2=-1 
\end{align}
solving the equation gives 0=2 \textrightarrow contradiction 
\end{enumerate}
Now for 3 equations system:
\[
\begin{bmatrix}
    1&2&2&2\\ 0&2&2&4\\ 0&0&0&0
\end{bmatrix} \rightarrow \text{this is the echelon form, but there are pivots for x,y and not for z}
\]
\[
\begin{cases}
    x+y+2x=2 \\ 2y+2z=4
\end{cases} \rightarrow \text{any value of z would work}\ \rightarrow \text{there are infinite solutions.}\ 
\]
\begin{note}
    when we don't find a pivot for one of the variables we call it"the pivot-free variable" and the others are "the pivot variables".
\end{note}

\begin{enumerate}
    \item   
    $\begin{bmatrix}
        \blacksquare & \square & \square& \square\\
        \triangle & \blacksquare & \square& \square\\
        \triangle & \triangle &   \blacksquare& \square
    \end{bmatrix}$ \textrightarrow unique solution 
    \item 
     $\begin{bmatrix}
        \blacksquare & \square & \square& \square\\
        \triangle & \blacksquare & \square& \square\\
        \triangle & \triangle &\triangle & \triangle 
    \end{bmatrix}$ \textrightarrow infinite solution 
    \item
     $\begin{bmatrix}
        \blacksquare & \square & \square& \square\\
        \triangle & \blacksquare & \square& \square\\
        \triangle & \triangle &  \triangle & \blacksquare
    \end{bmatrix}$ \textrightarrow contradiction(no solution )
\end{enumerate}
where $\blacksquare$ is any non-zero(pivot) and  $\square$ is any number including zero.
\subsection{Row echelon form:}
\begin{enumerate}
    \item All non-Zero rows are above zero rows.
    \item First number of a row is on the right of the first number (pivot) in the row above.
    \item All entries below each pivot are zero.
\end{enumerate}
Note:
\begin{itemize}
    \item If there are any free variables, that means that Now the system has infinite solutions.
    \item If there are any Contradictions, that means that the system has no solution.
    \item If all pivots are present (no. of pivots = of variables), then the system has one, unique solution.
    \item A system of equations is Consistent if it has any no of solutions. 
    \item A system of equations is inconsistent if
it has no solutions.
\end{itemize}
\begin{example}
    $\begin{cases}
        x_2-4x_3=8\\ 2x_1 -3x_2+2x_3=1 \\ 4x_1 -8x_2+12x_3=1 
    \end{cases} \rightarrow 
    \begin{bmatrix}
        0&1&-4&8 \\ 2&-3&2&1 \\ 4&-8&12&1 
    \end{bmatrix} \xrightarrow[R_2\leftrightarrow R_1]{-2R_2+R_3}
    \begin{bmatrix}
        2&-3&2&1 \\ 0&1&-4&8 \\ 0&-2&8&-1 
    \end{bmatrix}\\ \xrightarrow{2R_2+R_3}
    \begin{bmatrix}
        2&-3&2&1 \\ 0&1&-4&8 \\ 0&0&0&15 
    \end{bmatrix} \rightarrow 0=15$ Contradiction -no solution-.  
\end{example}
\begin{example}
    $\begin{bmatrix}
        1&1&2&2 \\ 1&1&2&3\\ 1&1&2&4
    \end{bmatrix}$\textrightarrow 3 parallel planes which mean that there is no solution. 
\end{example}
\begin{note}
    the ratio between entries above each other $\neq1$ always/ is not constant, so which means they are parallel, not coincident. 
\end{note}
\begin{example}
    $A \begin{bmatrix}
        1&1&2&2\\ 0&2&1&4\\ 0&0&0&3
    \end{bmatrix} \rightarrow \text{No solution}$ 
    $ B \begin{bmatrix}
        1&1&2&2\\ 0&2&1&4\\ 0&0&5&0
    \end{bmatrix} \rightarrow \text{Unique solution}$ \\
    $ C \begin{bmatrix}
        1&1&2&2&4\\ 0&2&1&4&4\\ 0&0&1&0&3 \\ 0&0&0&0&4
    \end{bmatrix} \rightarrow \text{No solution}$ 
     $ D \begin{bmatrix}
        1&1&2&2&4\\ 0&2&1&4&4\\ 0&0&0&0&3 \\ 0&0&0&0&0
    \end{bmatrix} \rightarrow \text{No solution}$ \\
     $ E \begin{bmatrix}
        1&1&2&2&4\\ 0&2&1&4&4\\ 0&0&0&0&4
    \end{bmatrix} \rightarrow \text{No solution}$ \\
     $  F \begin{bmatrix}
        1&1&2&2&4\\ 0&2&1&4&4\\ 0&0&0&0&0 \\ 0&0&0&0&0
    \end{bmatrix} \rightarrow \text{infinte solutions}\rightarrow \begin{cases}
        x_1+x_2+x_2+2x_3+2x_4=4\\ 2x_2+x_3+4x_4=4
    \end{cases}$ \\
     $ G \begin{bmatrix}
        1&1&2\\ 0&2&1\\ 0&0&0 
    \end{bmatrix} \rightarrow \text{Unique solution} \rightarrow 2x_2=1\rightarrow x_2 =\frac{1}{2} ,x_1=\frac{3}{2}$ \\
     $ H \begin{bmatrix}
        1&1&2\\ 0&2&1\\ 0&0&4
    \end{bmatrix} \rightarrow \text{No solution} \rightarrow \text{2 variables-contradiction-}$ 
\end{example}
the general solution for example F is \begin{itemize}
    \item $x_2= \frac{4-x_3-4x_4}{2}$
    \item $x_1=4-2x_3-2x_4-x_2=4-2x_3-2x_4- \frac{4-x_3-4x_4}{2}$
\end{itemize}
\end{document}

